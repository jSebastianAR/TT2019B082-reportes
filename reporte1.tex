\documentclass[10pt,letterpaper]{article}

\usepackage[spanish]{babel}

\begin{document}

\title{Titulo}
\author{%
Sergio Mercado Nu\~nez
\and
Jair Sebasti\'an Acosta Rosales
\and
Bella Citlalli Mart\'inez Seis
\and
Israel Buitr\'on D\'amaso
}
\date{\today}

\maketitle

\begin{abstract}
  Resumen del documento
\end{abstract}

\section{Introducción} % (fold)
\label{sec:introduccion}

\begin{itemize}
\item Acceder a cada una de las bases de datos.
\item Analizar el contenido de cada una. 
\item Detectar cuales campos son de mayor utilidad para una primera revisi\'on.
\item Filtrar la informaci\'on del Estado de inter\'es.
\item Entender cada uno de los campos que se est\'an seleccionando.
\end{itemize}	

%
% section introduccion (end)
%
\section{Datasets}

En primer instancia se cuenta con un conjunto de datasets con el que se decidi\'o dar inicio a la b\'usqueda de datos relacionados al problema que se desea resolver, de este modo, se tienen los siguientes conjuntos de datos para ser revisados:


\begin{enumerate}
\item \textbf{Programa de apoyo a la agricultura a trav\'es de insumos estrat\'egicos} 

Este Dataset cuenta con informaci\'on de insumos otorgados a personas en diferentes municipios del Estado de Jalisco, el total de registros que se tienen en este dataset es de 10639, la informaci\'on especificada en este es relacionada al nombre de cada uno de los beneficiarios del programa, as\'i como el Municipio y la Localidad a la que pertenecen, el concepto de apoyo en este caso sacos de semillas, la cantidad de sacos otorgada a cada beneficiario, as\'i como el monto total por persona de los apoyos entregados.

\item \textbf{Conjunto de datos vectoriales de la carta de uso del suelo y vegetaci\'on en Guadalajara}

Este conjunto de datos posee diferentes categorias de actividades que se reali\'an en el municipio de Guadalajara, tales actividades son: \textbf{Actividades pecuarias, Agricultura y vegetaci\'on, Cultivos, Especies Vegetales}.\\

\textit{Actividades Pecuarias} \\
Estos datos contienen la informaci\'on del tipo de ganado de animales que se poseen en ciertas zonas, como lo pueden ser Bovinos o Caprinos, as\'i como tambi\'en poseen las coordenadas de las locaciones, tanto en el sistema UTM, como en el sistema LCC. \\

\textit{Agricultura y vegetaci\'on}\\
Esta tabla es la que contiene m\'as informaci\'on de todo el dataset con aproximadamente 2302 registros de diferentes localidades en las cuales se describe la vegetaci\'on predominante as\'i como la vegetaci\'on secundaria existente, la erosi\'on del terreno, y tres diferentes apartados para cultivo: primario, secundario y terciario, donde se describe cada cu\'anto tiempo pueden ser cultivados, ya sea anual, permanentemente, semi-permanente e inclusive podr\'ia no aplicar o no tener ningun tipo de cultivo que pueda ser posible cultivar all\'i.\\

\textit{Cultivos}\\
Esta tabla contiene parejas datos acerca de cultivos en un lugar espec\'ifico, se nos describe el nombre de cada uno de los dos cultivos, as\'i como una respectiva clave \'unica usada para referirse al cultivo, tambi\'en se encuentran las coordenadas en el sistema UTM y LCC del punto de ubicaci\'on.

\textit{Especies Vegetales}\\
Al igual que la tabla de Cultivos, esta tabla contiene la informaci\'on de conjuntos vegetales que pueden ser cultivados en un mismo punto, la gran diferencia es que pueden ser hasta 5 especies vegetales en un solo punto.

\item \textbf{Mapas de uso del suelo y vegetacion Jalisco}\\
Este conjunto de datos pretende dar informaci\'on acerca del uso del suelo y de los recursos vegetales con los que se cuenta, contando con 6 tablas de datos, sin embargo el n\'umero de datos que posee cada una de ellas no supera los 5 registros, por lo que se cuentan con un total de 21 registros en total por todo el dataset, siendo este un dataset muy pobre en cuanto a contenido.
\end{enumerate}


\end{document}