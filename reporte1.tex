\documentclass[10pt,letterpaper]{article}

\usepackage[spanish]{babel}

\usepackage{xcolor-material}

\usepackage[todonotes={textsize=footnotesize}]{changes}
\definechangesauthor[name={Israel},color=MaterialBlue400]{IBD}
\definechangesauthor[name={Bella},color=MaterialIndigo300]{BCMS}
% \definechangesauthor[name={Sergio},color=<color>]{SMN}
% \definechangesauthor[name={Sebasti\'an},color=<color>]{JSAR}

\usepackage[utf8]{inputenc}

\PassOptionsToPackage{hyphens}{url}\usepackage{hyperref}

\setlength{\parskip}{1em}

\newcommand{\descripcion}{Un elemento dentro de esta categor\'ia cuenta con los siguientes atributos:}
\newcommand{\total}{Total de elementos: }
\newcommand{\UID}{\textbf{UID:} El identificador \'unico del elemento. Los ID \'unicos se pueden agrupar en funci\'on de un atributo com\'un.}
\newcommand{\OID}{\textbf{ObjectID:} Un Id. de objeto es un campo \'unico de enteros no nulos que se utiliza para identificar de forma \'unica las filas en las tablas de una geodatabase. En el caso que nos ocupa, el ObjectID y el UID tienen el mismo valor.}
\newcommand{\ent}{\textbf{Entidad:} Hace referencia a la categor\'ia del elemento }
\newcommand{\SHA}{\textbf{Shape\_area:} Es el c\'alculo del \'area plana para los pol\'igonos de una geodatabase anidada.}
\newcommand{\SHL}{\textbf{Shape\_length:} Es el c\'alculo del per\'imetro para los pol\'igonos de una geodatabase anidada.}
\newcommand{\GISA}{\textbf{GIS\_area:} contiene valores de \'area actualizados din\'amicamente para cada pol\'igono en una capa de pol\'igonos.}
\newcommand{\GISL}{\textbf{GIS\_length:} contiene valores de longitud actualizados din\'amicamente para cada l\'inea en una capa de polil\'inea.}

\begin{document}

\title{Titulo}
\author{%
Sergio Mercado Nu\~nez
\and
Jair Sebasti\'an Acosta Rosales
\and
Bella Citlalli Mart\'inez Seis
\and
Israel Buitr\'on D\'amaso
}
\date{\today}

\maketitle

\begin{abstract}
  Reporte interno del estudio de los primeros conjuntos de datos de inter\'es para este proyecto.
  Enlistamos las actividades realizadas en este estudio.
  Presentamos una descripci\'on breve de los conjuntos de datos estudiados.
\end{abstract}

\section{Introducci\'on} % (fold)
\label{sec:introduccion}

En este reporte se detalla la labor de estudio de los conjuntos de datos, considerados inicialmente, para ser posibles fuentes de informaci\'on.

Las actividades consideradas en este estudio son:
\begin{itemize}
\item Acceder a cada una de las bases de datos.
\item Analizar el contenido de cada una. 
\item Detectar cuales campos son de mayor utilidad para una primera revisi\'on.
\item Filtrar la informaci\'on del Estado de inter\'es.
\item Entender cada uno de los campos que se est\'an seleccionando.
\end{itemize}	

%
% section introduccion (end)
%

\section{Conjuntos de datos} % (fold)
\label{sec:conjuntos_de_datos}

En esta secci\'on se presentan los \emph{conjuntos de datos} (\emph{datasets}) con los que se inicia el an\'alisis para determinar la utilidad en este proyecto.

%\begin{enumerate}
%\item \textbf{Programa de apoyo a la agricultura a trav\'es de insumos estrat\'egicos} 
\subsection{Programa de apoyo a la agricultura a trav\'es de insumos estrat\'egicos}

Este conjunto de datos contiene la informaci\'on de insumos otorgados a personas en diferentes municipios del Estado de Jalisco.
Este contiene $10639$ registros.
La informaci\'on contenida est\'a relacionada al nombre del beneficiario, a su municipio y a su localidad de procedencia, al n\'umero de sacos suministrados y el monto total de apoyo otorgado.

%\item \textbf{Conjunto de datos vectoriales de la carta de uso del suelo y vegetaci\'on en Guadalajara}
\subsection{Conjunto de datos vectoriales de la carta de uso del suelo y vegetaci\'on en Guadalajara}

Este conjunto de datos posee diferentes categorias de actividades que se reali\'an en el municipio de Guadalajara, tales actividades son: \textbf{Actividades pecuarias, Agricultura y vegetaci\'on, Cultivos, Especies Vegetales}.\\

\textit{Actividades Pecuarias} \\
Estos datos contienen la informaci\'on del tipo de ganado de animales que se poseen en ciertas zonas, como lo pueden ser Bovinos o Caprinos, as\'i como tambi\'en poseen las coordenadas de las locaciones, tanto en el sistema UTM, como en el sistema LCC. \\

\textit{Agricultura y vegetaci\'on}\\
Esta tabla es la que contiene m\'as informaci\'on de todo el dataset con aproximadamente 2302 registros de diferentes localidades en las cuales se describe la vegetaci\'on predominante as\'i como la vegetaci\'on secundaria existente, la erosi\'on del terreno, y tres diferentes apartados para cultivo: primario, secundario y terciario, donde se describe cada cu\'anto tiempo pueden ser cultivados, ya sea anual, permanentemente, semi-permanente e inclusive podr\'ia no aplicar o no tener ningun tipo de cultivo que pueda ser posible cultivar all\'i.\\

\textit{Cultivos}\\
Esta tabla contiene parejas datos acerca de cultivos en un lugar espec\'ifico, se nos describe el nombre de cada uno de los dos cultivos, as\'i como una respectiva clave \'unica usada para referirse al cultivo, tambi\'en se encuentran las coordenadas en el sistema UTM y LCC del punto de ubicaci\'on.

\textit{Especies Vegetales}\\
Al igual que la tabla de Cultivos, esta tabla contiene la informaci\'on de conjuntos vegetales que pueden ser cultivados en un mismo punto, la gran diferencia es que pueden ser hasta 5 especies vegetales en un solo punto.

%\item \textbf{Mapas de uso del suelo y vegetacion Jalisco}\\
\subsection{Mapas de uso del suelo y vegetacion Jalisco}

Este conjunto de datos pretende dar informaci\'on acerca del uso del suelo y de los recursos vegetales con los que se cuenta, contando con 6 tablas de datos, sin embargo el n\'umero de datos que posee cada una de ellas no supera los 5 registros, por lo que se cuentan con un total de 21 registros en total por todo el dataset, siendo este un dataset muy pobre en cuanto a contenido.
%\end{enumerate}

%
% section conjuntos_de_datos (end)
%

\subsection{Conjunto de datos vectoriales de la serie topogr\'afica escala 1:1 000 000}

Contienen la informaci\'on sobre los diversos rasgos geogr\'aficos presentes en la Carta Topogr\'afica. Estos rasgos son representados digitalmente por un componente geom\'etrico (puntos, l\'ineas o \'areas), y un componente descriptivo (los atributos del rasgo). La informaci\'on se obtuvo derivada de los Conjuntos de Datos Topogr\'aficos 1:250,000, serie II, actualizados de noviembre de 1995 a noviembre de 1997. Per\'iodo de tiempo del Conjunto de Datos: enero de 1993 a abril de 1997. El tratamiento para generalizaci\'on de datos de la escala 1:250,000 a 1:1'000,000 se realiz\'o de enero a julio de 2000.\\

Enlace al Conjunto de Datos: 
\url{https://datos.gob.mx/busca/dataset/conjunto-de-datos-vectoriales-de-informacion-topografica-escala-1-1000-000}

Los datos topogr\'aficos disponibles en el conjunto de datos se encuentran divididos en las siguientes categor\'ias:
\begin{enumerate}
	\item \textbf{cuerposagua}
	\item \textbf{aeropuerto}
	\item \textbf{acueducto}
	\item \textbf{viaferrea}
	\item \textbf{vegetaciondensa}
	\item \textbf{rutaembarcacion}
	\item \textbf{puente}
	\item \textbf{plantageneradora}
	\item \textbf{mex\_toponloc\_i92}
	\item \textbf{mex\_nomgeo\_i92}
	\item \textbf{localidadurbana}
	\item \textbf{lineatransmision}
	\item \textbf{faro}
	\item \textbf{entradagruta}
	\item \textbf{curvasnivel}
\end{enumerate}

\subsubsection{Cuerpos de agua}
Extensi\'on de agua limitada por tierra.\descripcion
\begin{itemize}
	\item \UID
	\item \OID
	\item \ent (cuerpo de agua).
	\item \textbf{Tipo:} Se refiere al tipo de cuerpo de agua. Se manejan dos tipos:
		\begin{itemize}
			\item[--] \textbf{Intermitente:} Con presencia de agua en determinadas \'epocas del a\~no.
			\item[--] \textbf{Perenne:} Con presencia de agua permanentemente.
		\end{itemize}
	\item \SHA
	\item \SHL
	\item \GISA \\
	Las \'areas (pol\'igonos) son una serie de coordenadas geogr\'aficas unidas para formar un l\'imite. Una \'area es una l\'inea cerrada. Una entidad de \'area tiene una longitud y un ancho y puede tener atributos. En \textbf{GIS} (Geographic Information System: es un sistema dise\~nado para capturar, almacenar, manipular, analizar, administrar y presentar todo tipo de datos geogr\'aficos) las caracter\'isticas del \'area se denominan pol\'igonos. Un pol\'igono es un arco \'unico o una serie de arcos que est\'an conectados entre s\'i para encerrar un \'area.
	\item \GISL
\end{itemize}
\total \textbf{1098.}

\subsubsection{Puentes}
Estructuras que permiten el paso de una v\'ia de comunicaci\'on terrestre sobre un obst\'aculo natural o artificial. \descripcion
\begin{itemize}
	\item \UID
	\item \OID
	\item \textbf{Condici\'on:} Indica si el puente est\'a en operaci\'on, en construcci\'on o fuera de uso. El \'unico valor disponible entre los datos es \emph{En operaci\'on}.
	\item \ent (puente).
	\item \SHL
	\item \GISL
\end{itemize}
\total \textbf{126.}

\subsubsection{Acueductos}
Conductos artificiales empleados para transportar agua potable. \descripcion
\begin{itemize}
	\item \UID
	\item \OID
	\item \ent (acueducto).
	\item \textbf{RelSuelo:} Este atributo tiene por \'unico valor \emph{Subterr\'aneo} para cada uno de los elementos en la categor\'ia; el cual describe el tipo de acueducto.
	\item \SHL
	\item \GISL
\end{itemize}
\total \textbf{917.}

\subsubsection{V\'ias F\'erreas}
V\'ias de comunicaci\'on terrestre, cuya estructura consta de una terrapl\'en y dos rieles fijados mediante durmientes, para el tr\'ansito de trenes. \descripcion
\begin{itemize}
	\item \UID
	\item \OID
	\item \textbf{Tipo:} Hace referencia a la clase de v\'ia disponible en una zona; hay tres valores disponibles que puede adquirir ese atributo:
	\begin{itemize}
		\item[--] \textbf{V\'ia sencilla:} Que solo tiene una v\'ia en toda su longitud y por ella se verifica el movimiento de los trenes en ambos sentidos, ejecut\'andose el cruce de los mismos en las estaciones y algunos puntos determinados, donde se sit\'uan con tal objeto v\'ias dobles o apartaderos (un desv\'io o cambio de agujas es un aparato de v\'ia que permite a los trenes cambiar de una v\'ia a otra).
		\item[--] \textbf{V\'ia doble:} Que en toda su longitud tiene la v\'ia doble, dedic\'andose cada una para la marcha de los trenes en un sentido.
		\item[--] \textbf{N/A:} No aplicable. Existen 92 elementos con este valor, de los 2424 disponibles. 
	\end{itemize}
	\item \textbf{Condici\'on:} Indica si la v\'ia est\'a operativa o no. El \'unico valor disponible entre los datos es \emph{En operaci\'on}.
	\item \ent (v\'ia f\'errea).
	\item \SHL
	\item \GISL
\end{itemize}
\total \textbf{2424.}

\subsubsection{Vegetaci\'on Densa}
\descripcion
\begin{itemize}
	\item \UID
	\item \OID
	\item \ent (vegetaci\'on densa).
	\item \SHA
	\item \SHL
	\item \GISA
	\item \GISL
\end{itemize}
\total \textbf{868.}

\subsubsection{Rutas de embarcaci\'on}
Rutas sobre el agua que siguen las embarcaciones que regularmente transportan veh\'iculos y pasajeros. \descripcion
\begin{itemize}
	\item \UID
	\item \OID
	\item \textbf{Tipo:} Se indica que tipo de embarcaciones transitan por la ruta:
	\begin{itemize}
		\item[--] \textbf{Chalana:} Embarcaci\'on de fondo plano que se utiliza para el transporte de pasajeros y autom\'oviles de una orilla a otra de un r\'io.
		\item[--] \textbf{Panga:} Embarcaci\'on de fondo plano que se utiliza para el transporte de pasajeros y autom\'oviles de una orilla a otra de un r\'io, \'esta embarcaci\'on es de dimensiones menores que la chalana.
		\item[--] \textbf{Transbordador:} Barco acondicionado para transportar de un puerto a otro autom\'oviles, vagones, pasajeros, etc. 
	\end{itemize}
	\item \ent (ruta de embarcaci\'on).
	\item \SHL
	\item \GISL
\end{itemize}
\total \textbf{20.}

%\subsubsection{Rocas}
%Estructura escarpada que emerge de la superficie del mar. \descripcion
%\begin{itemize}
%	\item \UID
%	\item \OID	
%\end{itemize}

\subsubsection{Plantas generadoras}
Instalaciones para producir energ\'ia el\'ectrica. \descripcion
\begin{itemize}
	\item \UID
	\item \OID
	\item \ent (planta generadora).
	\item \textbf{Tipo de Planta Generadora:} 
	\begin{itemize}
		\item[--] \textbf{Geot\'ermica:} Por acci\'on del vapor de agua generado en el interior de la tierra.
		\item[--] \textbf{Hidroel\'ectrica:} Por acci\'on de fuerza hidr\'aulica.
		\item[--] \textbf{Nucleoel\'ectrica:} Por acci\'on del vapor de agua generado mediante el uso de la energ\'ia nuclear.
		\item[--] \textbf{Termoel\'ectrica:} Por acci\'on del vapor de agua.
	\end{itemize}
	\item \textbf{Condici\'on de la Planta Generadora:}
	\begin{itemize}
		\item[--] \textbf{En construcci\'on:} Que est\'a en un proceso de construcci\'on.
		\item[--] \textbf{En operaci\'on:} Que est\'a en servicio o puede usarse.
	\end{itemize}
\end{itemize}
\total \textbf{80.}

\subsubsection{L\'ineas de transmisi\'on}
Conjunto de cables a\'ereso, empleados para conducci\'on de energ\'ia el\'ectrica. \descripcion
\begin{itemize}
	\item \UID
	\item \OID
	\ent (l\'inea de transmisi\'on)
	\item \textbf{Alineamiento:} N\'umero de alineamientos de soporte.
	\begin{itemize}
		\item[--] \textbf{Una l\'inea de torres de acero}.
		\item[--] \textbf{Dos l\'ineas de torres de acero}.
		\item[--] \textbf{M\'as de dos l\'ineas de torres de acero}.
	\end{itemize}
	\item \SHL
	\item \GISL
\end{itemize}
\total \textbf{2423.}

\subsubsection{Curvas de nivel}
L\'ineas imaginarias que une puntos con la misma elevaci\'on con respecto al nivel del mar empleada para representar el relieve del terreno. \descripcion
\begin{itemize}
	\item \UID
	\item \OID
	\item \ent (curva de nivel).
	\item \textbf{Tipo de curva de nivel:}
	\begin{itemize}
		\item[--] \textbf{Depresi\'on:} Para representar un hundimiento en el terreno donde no hay salida del drenaje.
		\item[--] \textbf{Otros:} Cualquiera diferente al definido.
	\end{itemize}
	\item \textbf{Elevaci\'on de la curva de nivel:} Valor en metros de la elevaci\'on. De 0 a 5600.
	\item \textbf{Length:} Longitud.
	\item \SHL
	\item \GISL
\end{itemize}
\total \textbf{99488.}

\subsection{Valor de la producci\'on agr\'icola del Estado de Jalisco}

Informaci\'on del valor de la producci\'on agr\'icola del Estado de Jalisco y sus municipios. Presenta cifras expresadas en miles de pesos. Las cifras presentadas van del a\~no 2003 al 2014. Se incluye la informaci\'on de 123 de los 125 municipios del estado de Jalisco (los municipios excluidos son Guadalajara y San Ignacio Cerro Gordo).

Enlace al Conjunto de Datos:
\url{https://datos.gob.mx/busca/dataset/valor-produccion-agricola-jalisco}

\subsection{Peligros y sistemas afectables para el municipio de zapotlan el grande}
Este dataset contiene 18 tablas de información acerca de los cultivos y diferentes factores de riesgo para estos en la zona de zapotlan el grande, las tablas son las siguientes:  \\

\begin{enumerate}
\item \textbf{14023\_AGRICULTURA}\\

En esta tabla de datos, podemos encontrar dos datos interesantes acerca de los cultivos, el primero es el tipo de cultivo, ya sea aguacate, jitomate, frambuesa, zarzamora, granada, ar\'andano, fresa, agave, ma\'iz, alfalfa, sorgo, y adem\'as un registro de los que es un terreno agostadero, es decir, un terreno dedicado para que el ganado paste.

El segundo campo importante se refiere a las h\'ectareas ocupadas por cada uno de los sembrad\'ios que han sido registrados.

\item \textbf{14023\_AGRICULTURA\_PROTEGIDA}\\

Esta base de datos contiene algunos de los registros de la tabla anterior, e igualmente incluyen el nombre del cultivo y las h\'ectareas que ocupan, con la diferencia de que solo est\'an los registros de zonas que son protegidas.

\item \textbf{14023\_Dep\_cenz}\\

Contiene tres registros referidos a la actividad v\'olcanica, donde se evalua como medio, alto o muy alto cada uno de los registros, dando tambi\'en una breve descripci\'on del \'area(en km) que llega a ocupar la ceniza volc\'anica.

\item \textbf{14023\_Ero\_recep}\\

Contiene dos registros que describen la sedimentaci\'on, encontrandose en muy alto y muy bajo cada uno de ellos.

\item \textbf{14023\_Ero\_tipo}\\

Esta tabla contiene registros acerca de el tipo de erosi\'on que contiene algunos terrenos, as\'i como el grado de intensidad en que se ha clasificado pudiendo ser bajo, medio, alto o muy alto.

\item \textbf{14023\_Fractura}\\

Contiene registros que hablan acerca de las fracturas en el campo y la distancia que estas abarcan as\'i como una clasificaci\'on del grado de intensidad de las mismas.

\item \textbf{14023\_Hel\_cult}\\

En esta tabla de datos podemos ver la temperatura indicada en la que los cultivos(descritos en la tabla 14023\_AGRICULTURA) no se ven afectados en su producci\'on indicando cual es el l\'imite de temperaturas bajas que puede aguantar y/o indicando un rango entre las temperaturas m\'as adecuadas para su cultivo.

\item \textbf{14023\_Hel\_grdo}\\

La tabla describe algunas de las consecuencias que las bajas temperaturas tienen sobre los cultivos.

\item \textbf{14023\_Hur\_Isobar}\\

Describe el cambio de clima provocado por ciclones tropicales de diferentes tipos como lo puede ser: depresi\'on tropical, tormenta tropical o un hurac\'an.

\item \textbf{14023\_Inundacion}\\

En esta tabla de datos hay registros de inundaciones, as\'i como la profundidad de cada una de estas, aunque no se especif\'ica la zona en donde estas suceden.

\item \textbf{14023\_Nev\_grdo}\\

Esta tabla define en qu\'e temporadas hay m\'as probabilidad de que haya nevadas, tomando como base el nivel del mar sobre la cual se presentan.

\item \textbf{14023\_Seq\_cult}\\

Esta tabla contiene registros de la cantidad de agua(en mm) necesaria para poder mantener los cultivos(descritos en la tabla 14023\_AGRICULTURA) en buen estado, sin embargo existen otros registros que describen que se debe tener un riego intensivo en los cultivos m\'as no la cantidad(mm) del riego.

\item \textbf{14023\_Seq\_Veg}\\

En esta tabla se registran aquellos terrenos que cuentan con vegetaci\'on secundaria, as\'i como el tipo de vegetaci\'on que es y el nivel de intensidad en que se presenta.

\item \textbf{14023\_Tor\_elec\_grdo}\\

Especifica la densidad de impactos a tierra(cantidad de impactos de rayo en la tierra) por Km2, as\'i como la cantidad de d\'ias promedio que se tienen de tormenta el\'ectrica.

\item \textbf{14023\_Viento\_Isota}\\

Los registros de esta tabla se refieren a los diferentes tipos de vientos que hay en la zona, as\'i como sus velocidades y posibles da\~nos que pueden ocasionar.

\item \textbf{14023\_Volc\_act}\\

Describe \'areas donde hay peligro de actividad v\'olcanica y si se permite realizar actividades econ\'omicas en dichas zonas o esta totalmente prohibido.

\item \textbf{14023\_Zon\_sis}\\

Describe la zonas s\'ismicas poniendo la magnitud que se tiene en ellas en escala Richter.

\item \textbf{cobertura\_vegetal\_zapotlan}\\

Describe las diferentes \'areas del lugar, como lo pueden ser los dedicados a agricultura, zonas urbanas, bosques de tipos como mes\'ofilo de montaña, de oyamel, pino, pino y encino, encino, caducifolio, pastizal o pradera de montaña, también se describen cuerpos de agua, zonas sin vegetaci\'on aparente, y otras con vegetaci\'on secundaria como bosques tropicales de caducifolio, de pino y encino.

Junto a cada una de estas zonas tambi\'en se describe el \'area que abarca cada una de ellas.

\end{enumerate}



\end{document}
