\documentclass[10pt,letterpaper]{article}

\usepackage[spanish]{babel}

\usepackage{xcolor-material}

\usepackage[todonotes={textsize=footnotesize}]{changes}
\definechangesauthor[name={Israel},color=MaterialBlue400]{IBD}
\definechangesauthor[name={Bella},color=MaterialIndigo300]{BCMS}
% \definechangesauthor[name={Sergio},color=<color>]{SMN}
% \definechangesauthor[name={Sebasti\'an},color=<color>]{JSAR}

\usepackage[utf8]{inputenc}

\PassOptionsToPackage{hyphens}{url}\usepackage{hyperref}

\setlength{\parskip}{1em}

\newcommand{\descripcion}{Un elemento dentro de esta categoría cuenta con los siguientes atributos:}
\newcommand{\total}{Total de elementos: }
\newcommand{\UID}{\textbf{UID:} El identificador único del elemento. Los ID únicos se pueden agrupar en función de un atributo común.}
\newcommand{\OID}{\textbf{ObjectID:} Un Id. de objeto es un campo único de enteros no nulos que se utiliza para identificar de forma única las filas en las tablas de una geodatabase. En el caso que nos ocupa, el ObjectID y el UID tienen el mismo valor.}
\newcommand{\ent}{\textbf{Entidad:} Hace referencia a la categoría del elemento }
\newcommand{\SHA}{\textbf{Shape\_area:} Es el cálculo del área plana para los polígonos de una geodatabase anidada.}
\newcommand{\SHL}{\textbf{Shape\_length:} Es el cálculo del perímetro para los polígonos de una geodatabase anidada.}
\newcommand{\GISA}{\textbf{GIS\_area:} contiene valores de área actualizados dinámicamente para cada polígono en una capa de polígonos.}
\newcommand{\GISL}{\textbf{GIS\_length:} contiene valores de longitud actualizados dinámicamente para cada línea en una capa de polilínea.}

\begin{document}

\title{Titulo}
\author{%
Sergio Mercado Nu\~nez
\and
Jair Sebasti\'an Acosta Rosales
\and
Bella Citlalli Mart\'inez Seis
\and
Israel Buitr\'on D\'amaso
}
\date{\today}

\maketitle

\begin{abstract}
  Reporte interno del estudio de los primeros conjuntos de datos de inter\'es para este proyecto.
  Enlistamos las actividades realizadas en este estudio.
  Presentamos una descripci\'on breve de los conjuntos de datos estudiados.
\end{abstract}

\section{Introducción} % (fold)
\label{sec:introduccion}

En este reporte se detalla la labor de estudio de los conjuntos de datos, considerados inicialmente, para ser posibles fuentes de información.

Las actividades consideradas en este estudio son:
\begin{itemize}
\item Acceder a cada una de las bases de datos.
\item Analizar el contenido de cada una. 
\item Detectar cuales campos son de mayor utilidad para una primera revisi\'on.
\item Filtrar la informaci\'on del Estado de inter\'es.
\item Entender cada uno de los campos que se est\'an seleccionando.
\end{itemize}	

%
% section introduccion (end)
%

\section{Conjuntos de datos} % (fold)
\label{sec:conjuntos_de_datos}

En esta secci\'on se presentan los \emph{conjuntos de datos} (\emph{datasets}) con los que se inicia el an\'alisis para determinar la utilidad en este proyecto.

%\begin{enumerate}
%\item \textbf{Programa de apoyo a la agricultura a trav\'es de insumos estrat\'egicos} 
\subsection{Programa de apoyo a la agricultura a trav\'es de insumos estrat\'egicos}

Este conjunto de datos contiene la informaci\'on de insumos otorgados a personas en diferentes municipios del Estado de Jalisco.
Este contiene $10639$ registros.
La informaci\'on contenida est\'a relacionada al nombre del beneficiario, a su municipio y a su localidad de procedencia, al n\'umero de sacos suministrados y el monto total de apoyo otorgado.

%\item \textbf{Conjunto de datos vectoriales de la carta de uso del suelo y vegetaci\'on en Guadalajara}
\subsection{Conjunto de datos vectoriales de la carta de uso del suelo y vegetaci\'on en Guadalajara}

Este conjunto de datos posee diferentes categorias de actividades que se reali\'an en el municipio de Guadalajara, tales actividades son: \textbf{Actividades pecuarias, Agricultura y vegetaci\'on, Cultivos, Especies Vegetales}.\\

\textit{Actividades Pecuarias} \\
Estos datos contienen la informaci\'on del tipo de ganado de animales que se poseen en ciertas zonas, como lo pueden ser Bovinos o Caprinos, as\'i como tambi\'en poseen las coordenadas de las locaciones, tanto en el sistema UTM, como en el sistema LCC. \\

\textit{Agricultura y vegetaci\'on}\\
Esta tabla es la que contiene m\'as informaci\'on de todo el dataset con aproximadamente 2302 registros de diferentes localidades en las cuales se describe la vegetaci\'on predominante as\'i como la vegetaci\'on secundaria existente, la erosi\'on del terreno, y tres diferentes apartados para cultivo: primario, secundario y terciario, donde se describe cada cu\'anto tiempo pueden ser cultivados, ya sea anual, permanentemente, semi-permanente e inclusive podr\'ia no aplicar o no tener ningun tipo de cultivo que pueda ser posible cultivar all\'i.\\

\textit{Cultivos}\\
Esta tabla contiene parejas datos acerca de cultivos en un lugar espec\'ifico, se nos describe el nombre de cada uno de los dos cultivos, as\'i como una respectiva clave \'unica usada para referirse al cultivo, tambi\'en se encuentran las coordenadas en el sistema UTM y LCC del punto de ubicaci\'on.

\textit{Especies Vegetales}\\
Al igual que la tabla de Cultivos, esta tabla contiene la informaci\'on de conjuntos vegetales que pueden ser cultivados en un mismo punto, la gran diferencia es que pueden ser hasta 5 especies vegetales en un solo punto.

%\item \textbf{Mapas de uso del suelo y vegetacion Jalisco}\\
\subsection{Mapas de uso del suelo y vegetacion Jalisco}

Este conjunto de datos pretende dar informaci\'on acerca del uso del suelo y de los recursos vegetales con los que se cuenta, contando con 6 tablas de datos, sin embargo el n\'umero de datos que posee cada una de ellas no supera los 5 registros, por lo que se cuentan con un total de 21 registros en total por todo el dataset, siendo este un dataset muy pobre en cuanto a contenido.
%\end{enumerate}

%
% section conjuntos_de_datos (end)
%

\subsection{Conjunto de datos vectoriales de la serie topográfica escala 1:1 000 000}

Contienen la información sobre los diversos rasgos geográficos presentes en la Carta Topográfica. Estos rasgos son representados digitalmente por un componente geométrico (puntos, líneas o áreas), y un componente descriptivo (los atributos del rasgo). La información se obtuvo derivada de los Conjuntos de Datos Topográficos 1:250,000, serie II, actualizados de noviembre de 1995 a noviembre de 1997. Período de tiempo del Conjunto de Datos: enero de 1993 a abril de 1997. El tratamiento para generalización de datos de la escala 1:250,000 a 1:1'000,000 se realizó de enero a julio de 2000.\\

Enlace al Conjunto de Datos: 
\url{https://datos.gob.mx/busca/dataset/conjunto-de-datos-vectoriales-de-informacion-topografica-escala-1-1000-000}

Los datos topográficos disponibles en el conjunto de datos se encuentran divididos en las siguientes categorías:
\begin{enumerate}
	\item \textbf{cuerposagua}
	\item \textbf{aeropuerto}
	\item \textbf{acueducto}
	\item \textbf{viaferrea}
	\item \textbf{vegetaciondensa}
	\item \textbf{rutaembarcacion}
	\item \textbf{puente}
	\item \textbf{plantageneradora}
	\item \textbf{mex\_toponloc\_i92}
	\item \textbf{mex\_nomgeo\_i92}
	\item \textbf{localidadurbana}
	\item \textbf{lineatransmision}
	\item \textbf{faro}
	\item \textbf{entradagruta}
	\item \textbf{curvasnivel}
\end{enumerate}

\subsubsection{Cuerpos de agua}
Extensión de agua limitada por tierra.\descripcion
\begin{itemize}
	\item \UID
	\item \OID
	\item \ent (cuerpo de agua).
	\item \textbf{Tipo:} Se refiere al tipo de cuerpo de agua. Se manejan dos tipos:
		\begin{itemize}
			\item[--] \textbf{Intermitente:} Con presencia de agua en determinadas épocas del año.
			\item[--] \textbf{Perenne:} Con presencia de agua permanentemente.
		\end{itemize}
	\item \SHA
	\item \SHL
	\item \GISA \\
	Las áreas (polígonos) son una serie de coordenadas geográficas unidas para formar un límite. Una área es una línea cerrada. Una entidad de área tiene una longitud y un ancho y puede tener atributos. En \textbf{GIS} (Geographic Information System: es un sistema diseñado para capturar, almacenar, manipular, analizar, administrar y presentar todo tipo de datos geográficos) las características del área se denominan polígonos. Un polígono es un arco único o una serie de arcos que están conectados entre sí para encerrar un área.
	\item \GISL
\end{itemize}
\total \textbf{1098.}

\subsubsection{Puentes}
Estructuras que permiten el paso de una vía de comunicación terrestre sobre un obstáculo natural o artificial. \descripcion
\begin{itemize}
	\item \UID
	\item \OID
	\item \textbf{Condición:} Indica si el puente está en operación, en construcción o fuera de uso. El único valor disponible entre los datos es \emph{En operación}.
	\item \ent (puente).
	\item \SHL
	\item \GISL
\end{itemize}
\total \textbf{126.}

\subsubsection{Acueductos}
Conductos artificiales empleados para transportar agua potable. \descripcion
\begin{itemize}
	\item \UID
	\item \OID
	\item \ent (acueducto).
	\item \textbf{RelSuelo:} Este atributo tiene por único valor \emph{Subterráneo} para cada uno de los elementos en la categoría; el cual describe el tipo de acueducto.
	\item \SHL
	\item \GISL
\end{itemize}
\total \textbf{917.}

\subsubsection{Vías Férreas}
Vías de comunicación terrestre, cuya estructura consta de una terraplén y dos rieles fijados mediante durmientes, para el tránsito de trenes. \descripcion
\begin{itemize}
	\item \UID
	\item \OID
	\item \textbf{Tipo:} Hace referencia a la clase de vía disponible en una zona; hay tres valores disponibles que puede adquirir ese atributo:
	\begin{itemize}
		\item[--] \textbf{Vía sencilla:} Que solo tiene una vía en toda su longitud y por ella se verifica el movimiento de los trenes en ambos sentidos, ejecutándose el cruce de los mismos en las estaciones y algunos puntos determinados, donde se sitúan con tal objeto vías dobles o apartaderos (un desvío o cambio de agujas es un aparato de vía que permite a los trenes cambiar de una vía a otra).
		\item[--] \textbf{Vía doble:} Que en toda su longitud tiene la vía doble, dedicándose cada una para la marcha de los trenes en un sentido.
		\item[--] \textbf{N/A:} No aplicable. Existen 92 elementos con este valor, de los 2424 disponibles. 
	\end{itemize}
	\item \textbf{Condición:} Indica si la vía está operativa o no. El único valor disponible entre los datos es \emph{En operación}.
	\item \ent (vía férrea).
	\item \SHL
	\item \GISL
\end{itemize}
\total \textbf{2424.}

\subsubsection{Vegetación Densa}
\descripcion
\begin{itemize}
	\item \UID
	\item \OID
	\item \ent (vegetación densa).
	\item \SHA
	\item \SHL
	\item \GISA
	\item \GISL
\end{itemize}
\total \textbf{868.}

\subsubsection{Rutas de embarcación}
Rutas sobre el agua que siguen las embarcaciones que regularmente transportan vehículos y pasajeros. \descripcion
\begin{itemize}
	\item \UID
	\item \OID
	\item \textbf{Tipo:} Se indica que tipo de embarcaciones transitan por la ruta:
	\begin{itemize}
		\item[--] \textbf{Chalana:} Embarcación de fondo plano que se utiliza para el transporte de pasajeros y automóviles de una orilla a otra de un río.
		\item[--] \textbf{Panga:} Embarcación de fondo plano que se utiliza para el transporte de pasajeros y automóviles de una orilla a otra de un río, ésta embarcación es de dimensiones menores que la chalana.
		\item[--] \textbf{Transbordador:} Barco acondicionado para transportar de un puerto a otro automóviles, vagones, pasajeros, etc. 
	\end{itemize}
	\item \ent (ruta de embarcación).
	\item \SHL
	\item \GISL
\end{itemize}
\total \textbf{20.}

%\subsubsection{Rocas}
%Estructura escarpada que emerge de la superficie del mar. \descripcion
%\begin{itemize}
%	\item \UID
%	\item \OID	
%\end{itemize}

\subsubsection{Plantas generadoras}
Instalaciones para producir energía eléctrica. \descripcion
\begin{itemize}
	\item \UID
	\item \OID
	\item \ent (planta generadora).
	\item \textbf{Tipo de Planta Generadora:} 
	\begin{itemize}
		\item[--] \textbf{Geotérmica:} Por acción del vapor de agua generado en el interior de la tierra.
		\item[--] \textbf{Hidroeléctrica:} Por acción de fuerza hidráulica.
		\item[--] \textbf{Nucleoeléctrica:} Por acción del vapor de agua generado mediante el uso de la energía nuclear.
		\item[--] \textbf{Termoeléctrica:} Por acción del vapor de agua.
	\end{itemize}
	\item \textbf{Condición de la Planta Generadora:}
	\begin{itemize}
		\item[--] \textbf{En construcción:} Que está en un proceso de construcción.
		\item[--] \textbf{En operación:} Que está en servicio o puede usarse.
	\end{itemize}
\end{itemize}
\total \textbf{80.}

\subsubsection{Líneas de transmisión}
Conjunto de cables aéreso, empleados para conducción de energía eléctrica. \descripcion
\begin{itemize}
	\item \UID
	\item \OID
	\ent (línea de transmisión)
	\item \textbf{Alineamiento:} Número de alineamientos de soporte.
	\begin{itemize}
		\item[--] \textbf{Una línea de torres de acero}.
		\item[--] \textbf{Dos líneas de torres de acero}.
		\item[--] \textbf{Más de dos líneas de torres de acero}.
	\end{itemize}
	\item \SHL
	\item \GISL
\end{itemize}
\total \textbf{2423.}

\subsubsection{Curvas de nivel}
Líneas imaginarias que une puntos con la misma elevación con respecto al nivel del mar empleada para representar el relieve del terreno. \descripcion
\begin{itemize}
	\item \UID
	\item \OID
	\item \ent (curva de nivel).
	\item \textbf{Tipo de curva de nivel:}
	\begin{itemize}
		\item[--] \textbf{Depresión:} Para representar un hundimiento en el terreno donde no hay salida del drenaje.
		\item[--] \textbf{Otros:} Cualquiera diferente al definido.
	\end{itemize}
	\item \textbf{Elevación de la curva de nivel:} Valor en metros de la elevación. De 0 a 5600.
	\item \textbf{Length:} Longitud.
	\item \SHL
	\item \GISL
\end{itemize}
\total \textbf{99488.}

\subsection{Valor de la producción agrícola del Estado de Jalisco}

Información del valor de la producción agrícola del Estado de Jalisco y sus municipios. Presenta cifras expresadas en miles de pesos. Las cifras presentadas van del año 2003 al 2014. Se incluye la información de 123 de los 125 municipios del estado de Jalisco (los municipios excluidos son Guadalajara y San Ignacio Cerro Gordo).

Enlace al Conjunto de Datos:
\url{https://datos.gob.mx/busca/dataset/valor-produccion-agricola-jalisco}

\end{document}
